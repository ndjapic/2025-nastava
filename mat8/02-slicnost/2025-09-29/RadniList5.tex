\documentclass[11pt,a5paper,addpoints,noanswers]{exam}
%\documentclass[11pt,a5paper,addpoints,answers]{exam}

\usepackage{multicol}
\usepackage{amssymb,amsmath}
\usepackage{tikz} % Потребно за конструктивне задатке у решењима
\usepackage{graphicx,geometry}
\geometry{a5paper, margin=1.5cm}

\usepackage[OT2]{fontenc}
\usepackage[utf8x]{inputenc}
\usepackage[serbian]{babel}

\addto{\captionsserbian}{\renewcommand{\abstractname}{Упут{}ство}}

\renewcommand{\solutiontitle}{\noindent\textrm{Решење:}\enspace}
\pointsinmargin
\pointname{}
\hqword{Задатак:}
\hpgword{Страница:}
\hpword{Поени:}
\hsword{Остварено:}
\htword{Збир}
\cellwidth{1em}
\gradetablestretch{1.1}
\cfoot[]{Страница \thepage\ од \numpages}
\def\measure#1#2{#1 \, \mathrm{#2}}

\title{$\mathrm{VIII}_\Box$, група А (Основни ниво)}
\author{Сличност и раније градиво}
\date{29. септембар 2025.}

\def\abs|#1|{\left| #1 \right|}

\begin{document}

\maketitle
\thispagestyle{headandfoot}

\begin{center} \small
 \hrulefill \par (име и презиме ученика)
\end{center}

\begin{abstract}
Задаци су основног нивоа и носе различит број бодова (укупно 24 бода). Пажљиво прочитајте сваки задатак. Време за израду је 40 минута. Дозвољена је употреба лењира и шестара.
За добијање бодова, у сваком задатку мора бити наведен поступак из којег се јасно види како се дошло до решења.

\medskip
\noindent *Бодовање за оцену:* Недовољан (1) $0 - 6$ бодова, Довољан (2) $7 - 11$ бодова, Добар (3) $12 - 15$ бодова, Врло добар (4) $16 - 19$ бодова, Одличан (5) $20 - 24$ бода.
\end{abstract}

%\vspace*{\stretch 1}
\noindent \gradetable[h]

\begin{questions}

\question[3]
 Два угла једног троугла су
 $75^\circ$ и
 $45^\circ$, а два угла другог
 $60^\circ$ и
 $45^\circ$.
 Испитај да ли су та два троугла слична.
 \begin{solution}[\stretch 4] % Основни задатак о сличности - УУ став
  Троугао $\Delta ABC$:
  $\gamma_1 = 180^\circ - (75^\circ + 45^\circ) = 180^\circ - 120^\circ = 60^\circ$.
  Углови првог троугла су: $\{75^\circ, 45^\circ, 60^\circ\}$.

  Троугао $\Delta A'B'C'$:
  Трећи угао је: $\gamma_2 = 180^\circ - (60^\circ + 45^\circ) = 180^\circ - 105^\circ = 75^\circ$.
  Углови другог троугла су: $\{60^\circ, 45^\circ, 75^\circ\}$.

  Како су углови првог троугла једнаки угловима другог троугла (поређани: $75^\circ=75^\circ$, $60^\circ=60^\circ$, $45^\circ=45^\circ$), троуглови су слични по ставу У-У.
 \end{solution}
 \answerline

\ifprintanswers\else\newpage\fi % Први обавезни прелазак на нову страницу (страница 2)

\question[3]
 Дужине страница једног троугла су
 $\measure{3}{cm}$,
 $\measure{4}{cm}$ и
 $\measure{5}{cm}$, а другог
 $\measure{9}{cm}$,
 $\measure{12}{cm}$ и
 $\measure{15}{cm}$.
 Испитај да ли су та два троугла слична.
 \begin{solution}[\stretch 4] % Основни задатак о сличности - ССС став
  Странице првог троугла: $a_1=3$, $b_1=4$, $c_1=5$.
  Странице другог троугла: $a_2=9$, $b_2=12$, $c_2=15$.

  Проверавамо однос одговарајућих страница:
  $$ \frac{3}{9} = \frac{1}{3} $$
  $$ \frac{4}{12} = \frac{1}{3} $$
  $$ \frac{5}{15} = \frac{1}{3} $$
  Како је однос свих страница једнак $k = 1/3$, троуглови су слични по ставу С-С-С.
 \end{solution}
 \answerline

\question[3]
 Израчунај обим једнакокраког трапеза чије су основице $a=\measure{10}{cm}$ и $b=\measure{4}{cm}$, а крак $c=\measure{5}{cm}$.
 \begin{solution}[\stretch 4] % Задатак из претходног градива (Трапез)
  Обим трапеза је збир дужина свих страница.
  $O = a + b + 2c$
  $O = 10\,\mathrm{cm} + 4\,\mathrm{cm} + 2 \cdot 5\,\mathrm{cm}$
  $O = 14\,\mathrm{cm} + 10\,\mathrm{cm}$
  $O = \measure{24}{cm}$.
  Обим трапеза је $\measure{24}{cm}$.
 \end{solution}
 \answerline

\ifprintanswers\else\newpage\fi % Други обавезни прелазак на нову страницу (страница 3)

\question[4]
 Дата је дуж $AB$ подељена тачком $C$ на два дела. Ако је $AC=\measure{5}{cm}$, $CB=\measure{10}{cm}$. Израчунај дужину дужи $CD=x$ тако да су дужи $AC$ и $CD$ пропорционалне дужима $CB$ и $x$ (тј. $AC:CB=CD:x$), ако је $CD=\measure{2}{cm}$.
 \begin{solution}[\stretch 7] % Основни задатак пропорције (Талесова теорема)
  Постављамо пропорцију:
  $$ AC:CB = CD:x $$
  Убацујемо дате вредности:
  $$ 5:10 = 2:x $$
  Производ спољашњих чланова једнак је производу унутрашњих:
  $$ 5 \cdot x = 10 \cdot 2 $$
  $$ 5x = 20 $$
  $$ x = \frac{20}{5} $$
  $$ x = \measure{4}{cm} $$
  Дужина дужи $x$ је $\measure{4}{cm}$.
 \end{solution}
 \answerline

\question[3]
 Израчунај дужину хипотенузе правоуглог троугла чије су катете $a=\measure{6}{cm}$ и $b=\measure{8}{cm}$.
 \begin{solution}[\stretch 4] % Задатак из претходног градива (Питагорина теорема)
  Користимо Питагорину теорему: $c^2 = a^2 + b^2$.
  $c^2 = 6^2 + 8^2$
  $c^2 = 36 + 64$
  $c^2 = 100$
  $c = \sqrt{100}$
  $c = \measure{10}{cm}$.
  Дужина хипотенузе је $\measure{10}{cm}$.
 \end{solution}
 \answerline

\ifprintanswers\else\newpage\fi % Трећи обавезни прелазак на нову страницу (страница 4)

\question[4]
 Конструиши симетралу датог угла $\alpha = 70^\circ$.
 \begin{solution}[\stretch 7] % Задатак из претходног градива (Основна конструкција)
  Поступак конструкције симетрале угла $\alpha$:
  \begin{enumerate}
      \item Нацртати угао $\alpha = 70^\circ$.
      \item Из темена угла $A$ нацртати лук произвољног полупречника $r$ који сече оба крака угла у тачкама $B$ и $C$.
      \item Из тачке $B$ нацртати лук полупречника $r$ (или другог, али једног) унутар угла.
      \item Из тачке $C$ нацртати лук истог полупречника $r$ који сече претходно нацртани лук у тачки $D$.
      \item Полуправа која полази из темена $A$ и пролази кроз тачку $D$ је тражена симетрала $s$.
  \end{enumerate}
  Симетрала дели угао на два једнака дела од $35^\circ$.
 \end{solution}
 \answerline

\question[4]
 Два слична троугла имају коефицијент сличности $k = \frac{2}{3}$. Ако је обим мањег троугла $O_1 = \measure{18}{cm}$, израчунај обим већег троугла $O_2$.
 \begin{solution}[\stretch 9] % Основни задатак о обиму и сличности
  Коефицијент сличности $k$ је једнак односу обима:
  $$ \frac{O_1}{O_2} = k $$
  Замењујемо дате вредности:
  $$ \frac{18}{O_2} = \frac{2}{3} $$
  Решавамо пропорцију:
  $$ 2 \cdot O_2 = 18 \cdot 3 $$
  $$ 2 \cdot O_2 = 54 $$
  $$ O_2 = \frac{54}{2} $$
  $$ O_2 = \measure{27}{cm} $$
  Обим већег троугла је $\measure{27}{cm}$.
 \end{solution}
 \answerline

\end{questions}

\end{document}
