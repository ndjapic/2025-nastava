%\documentclass[11pt,a5paper,addpoints,noanswers]{exam}
\documentclass[10pt,a5paper,addpoints,answers]{exam}

\usepackage{multicol}
\usepackage{amssymb,amsmath}
\usepackage{tikz} % Потребно за конструктивне задатке у решењима
\usepackage{graphicx,geometry}
\geometry{a5paper, margin=1.5cm}

\usepackage[OT2]{fontenc}
\usepackage[utf8x]{inputenc}
\usepackage[serbian]{babel}

\addto{\captionsserbian}{\renewcommand{\abstractname}{Упут{}ство}}

\renewcommand{\solutiontitle}{\noindent\textrm{Решење:}\enspace}
\pointsinmargin
\pointname{}
\hqword{Задатак:}
\hpgword{Страница:}
\hpword{Поени:}
\hsword{Остварено:}
\htword{Збир}
\cellwidth{1em}
\gradetablestretch{1.1}
\cfoot[]{Страница \thepage\ од \numpages}
\def\measure#1#2{#1 \, \mathrm{#2}}

\title{$\mathrm{VIII}_\Box$, група А}
\author{Сличност}
\date{29. септембар 2025.}

\def\abs|#1|{\left| #1 \right|}

\begin{document}

\maketitle
\thispagestyle{headandfoot}

\begin{center} \small
 \hrulefill \par (име и презиме ученика)
\end{center}

\begin{abstract}
Задаци су распоређени по тежини и носе различит број бодова (укупно 24 бода). Пажљиво прочитајте сваки задатак. *Задаци 5 и 6 су конструктивни.* Време за израду је 40 минута. Дозвољена је употреба лењира и шестара.
За добијање бодова, у сваком задатку мора бити наведен поступак из којег се јасно види како се дошло до решења.

\medskip
\noindent *Бодовање за оцену:* Недовољан (1) $0 - 6$ бодова, Довољан (2) $7 - 11$ бодова, Добар (3) $12 - 15$ бодова, Врло добар (4) $16 - 19$ бодова, Одличан (5) $20 - 24$ бода.
\end{abstract}

%\vspace*{\stretch 1}
\noindent \gradetable[h]

\begin{questions}

\question[3]
 Два угла једног троугла су
 $73^\circ$ и
 $33^\circ$, а два угла другог
 $74^\circ$ и
 $33^\circ$.
 Испитај да ли су та два троугла слична.
 \begin{solution}[\stretch 4] % Повећан простор за рад
  Троугао $\Delta ABC$:
  Збир углова у троуглу је $180^\circ$.
  $\gamma_1 = 180^\circ - (73^\circ + 33^\circ) = 180^\circ - 106^\circ = 74^\circ$.
  Углови првог троугла су: $\{73^\circ, 33^\circ, 74^\circ\}$.

  Троугао $\Delta A'B'C'$:
  Збир углова у троуглу је $180^\circ$.
  Трећи угао је: $\gamma_2 = 180^\circ - (74^\circ + 33^\circ) = 180^\circ - 107^\circ = 73^\circ$.
  Углови другог троугла су: $\{74^\circ, 33^\circ, 73^\circ\}$.

  Како су углови првог троугла једнаки угловима другог троугла (поређани су $73^\circ=73^\circ$, $33^\circ=33^\circ$, $74^\circ=74^\circ$), на основу става У-У (Угао-Угао),
  троуглови су слични.
 \end{solution}
 \answerline

\ifprintanswers\else\newpage\fi % Први обавезни прелазак на нову страницу (страница 2)

\question[3]
 Дужине страница једног троугла су
 $\measure{4}{cm}$,
 $\measure{5}{cm}$ и
 $\measure{6}{cm}$, а другог
 $\measure{6}{cm}$,
 $\measure{7{,}5}{cm}$ и
 $\measure{9}{cm}$.
 Испитај да ли су та два троугла слична.
 \begin{solution}[\stretch 4] % Повећан простор за рад
  Странице првог троугла: $a_1=4$, $b_1=5$, $c_1=6$.
  Странице другог троугла: $a_2=6$, $b_2=7{,}5$, $c_2=9$.

  Поређамо странице по дужини и проверимо однос (коефицијент сличности $k$):
  \begin{itemize}
      \item Најкраћа: $\frac{4}{6} = \frac{2}{3}$
      \item Средња: $\frac{5}{7{,}5} = \frac{50}{75} = \frac{2}{3}$
      \item Најдужа: $\frac{6}{9} = \frac{2}{3}$
  \end{itemize}
  Како је однос свих парова одговарајућих страница једнак: $\frac{4}{6} = \frac{5}{7{,}5} = \frac{6}{9} = \frac{2}{3}$, на основу става С-С-С (Страница-Страница-Страница),
  троуглови су слични.
 \end{solution}
 \answerline

\question[4]
 Странице једног троугла су
 $a = \measure{63}{cm}$,
 $b = \measure{72}{cm}$ и
 $c = \measure{54}{cm}$,
 а збир две дуже
 странице њему сличног троугла је
 $\measure{75}{cm}$.
 Израчунај дужине страница тог, њему сличног троугла.
 \begin{solution}[\stretch 8] % Повећан простор за рад
  Странице датог троугла: $63\,\mathrm{cm}$, $72\,\mathrm{cm}$, $54\,\mathrm{cm}$.
  Две дуже странице су $b=72\,\mathrm{cm}$ и $a=63\,\mathrm{cm}$.

  Нека су странице сличног троугла $a'$, $b'$, $c'$.
  Дато је: $a'+b' = 75\,\mathrm{cm}$.
  По дефиницији сличности: $\frac{a'}{a} = \frac{b'}{b} = \frac{c'}{c} = k$.

  За дужности страница важи: $a' = k \cdot a$ и $b' = k \cdot b$.
  Заменимо у дати збир:
  $k \cdot a + k \cdot b = 75$
  $k \cdot (a + b) = 75$
  $k \cdot (63 + 72) = 75$
  $k \cdot 135 = 75$
  $k = \frac{75}{135} = \frac{5 \cdot 15}{9 \cdot 15} = \frac{5}{9}$

  Рачунање дужина страница сличног троугла:
  \begin{itemize}
      \item $a' = k \cdot a = \frac{5}{9} \cdot 63 = 5 \cdot 7 = \measure{35}{cm}$
      \item $b' = k \cdot b = \frac{5}{9} \cdot 72 = 5 \cdot 8 = \measure{40}{cm}$
      \item $c' = k \cdot c = \frac{5}{9} \cdot 54 = 5 \cdot 6 = \measure{30}{cm}$
  \end{itemize}
  Дужине страница сличног троугла су $\measure{40}{cm}$, $\measure{35}{cm}$ и $\measure{30}{cm}$. (Провера: $40+35=75$).
 \end{solution}
 \answerline

\ifprintanswers\else\newpage\fi % Други обавезни прелазак на нову страницу (страница 3)

\question[4]
 Једно дрво баца сенку дужине
 $\measure{6}{m}$.
 Истовремено, сенка човека који стоји поред њега је
 $\measure{1{,}5}{m}$.
 Израчунај висину дрвета, ако је човек висок
 $\measure{1{,}8}{m}$.
 \begin{solution}[\stretch 5] % Повећан простор за рад
  Ситуација висине објеката и дужина њихових сенки у исто време представља сличне правоугле троуглове.
  $H_d$ - висина дрвета (тражи се)
  $S_d = \measure{6}{m}$ - сенка дрвета
  $H_c = \measure{1{,}8}{m}$ - висина човека
  $S_c = \measure{1{,}5}{m}$ - сенка човека

  Користимо пропорцију:
  $$ \frac{H_d}{S_d} = \frac{H_c}{S_c} $$

  Замењујемо познате вредности:
  $$ \frac{H_d}{6} = \frac{1{,}8}{1{,}5} $$

  Решавамо по $H_d$:
  $$ H_d = 6 \cdot \frac{1{,}8}{1{,}5} $$
  $$ H_d = 6 \cdot \frac{18}{15} $$
  $$ H_d = 6 \cdot 1{,}2 $$
  $$ H_d = \measure{7{,}2}{m} $$
  Висина дрвета је $\measure{7{,}2}{m}$.
 \end{solution}
 \answerline

\question[3]
 Конструиши дуж $x$ која је четврта пропорционала за дате дужи $a, b, c$, тако да је $a:b=c:x$. Узми да је дужина дужи $a=\measure{3}{cm}$, $b=\measure{5}{cm}$ и $c=\measure{4}{cm}$.
 \begin{solution}[\stretch 7] % Повећан простор за рад
  Поступак конструкције:
  \begin{enumerate}
      \item Нацртати две полуправе $p$ и $q$ које полазе из заједничке тачке $O$.
      \item На полуправу $p$ од тачке $O$ нанети редом дужи $a=\measure{3}{cm}$ и $b=\measure{5}{cm}$. Нека су крајње тачке $A$ и $B$. (Дакле, $OA=a$, $AB=b$).
      \item На полуправу $q$ од тачке $O$ нанети дуж $c=\measure{4}{cm}$. Нека је крајња тачка $C$. ($OC=c$).
      \item Спојити тачке $A$ и $C$.
      \item Кроз тачку $B$ конструисати праву $r$ паралелну дужи $AC$.
      \item Права $r$ пресеца полуправу $q$ у тачки $D$. Дуж $CD$ је тражена дуж $x$. ($CD=x$).
  \end{enumerate}
  Образложење: Из сличности троуглова $\Delta OAC$ и $\Delta OBD$ (због $AC || BD$) следи $OA:AB = OC:CD$, односно $a:b=c:x$.
 \end{solution}
 \answerline

\ifprintanswers\else\newpage\fi % Трећи обавезни прелазак на нову страницу (страница 4)

\question[3]
 Конструиши дуж $x$ која је геометријска средина дужи $a$ и $b$, тако да је $x = \sqrt{a \cdot b}$. Узми да је дужина дужи $a=\measure{4}{cm}$ и $b=\measure{6}{cm}$.
 \begin{solution}[\stretch 7] % Повећан простор за рад
  Поступак конструкције (коришћењем висине у правоуглом троуглу - Еуклидова теорема о висини):
  \begin{enumerate}
      \item Нацртати праву $p$ и на њој нанети дуж $c = a+b = 4\,\mathrm{cm} + 6\,\mathrm{cm} = 10\,\mathrm{cm}$.
      \item Означити тачке $P$ и $Q$ тако да је $PQ = a+b = 10\,\mathrm{cm}$. Тачка раздвајања дужи $a$ и $b$ на хипотенузи је $R$, тако да је $PR = a = 4\,\mathrm{cm}$ и $RQ = b = 6\,\mathrm{cm}$.
      \item Конструисати полукруг над пречником $PQ$.
      \item У тачки $R$ конструисати нормалу на дуж $PQ$.
      \item Тачка пресека нормале и полукруга је $S$. Дуж $RS$ је тражена дуж $x$.
  \end{enumerate}
  Образложење: Висина $h=x$ у правоуглом троуглу дели хипотенузу на одсечке $p=a$ и $q=b$. По Еуклидовој теореми о висини, $h^2 = p \cdot q$, односно $x^2 = a \cdot b$, па је $x = \sqrt{a \cdot b}$.
 \end{solution}
 \answerline

\question[4]
 У правоуглом троуглу висина дели хипотенузу на дужи дужине
 $p = \measure{4{,}5}{cm}$ и $q = \measure{8}{cm}$.
 Израчунај обим тог троугла.
 \begin{solution}[\stretch 9] % Измењено са [stretch 10] на [stretch 9]
  Користимо Еуклидове теореме. Нека је хипотенуза $c = p+q$, а катете $a$ и $b$.
  1. Израчунавање хипотенузе ($c$):
  $c = p + q = 4{,}5 + 8 = \measure{12{,}5}{cm}$.

  2. Израчунавање катете $a$ (по Еуклидовој теореми):
  $a^2 = p \cdot c$
  $a^2 = 4{,}5 \cdot 12{,}5$
  $a^2 = \frac{9}{2} \cdot \frac{25}{2} = \frac{225}{4}$
  $a = \sqrt{\frac{225}{4}} = \frac{15}{2} = \measure{7{,}5}{cm}$.

  3. Израчунавање катете $b$ (по Еуклидовој теореми):
  $b^2 = q \cdot c$
  $b^2 = 8 \cdot 12{,}5 = 100$
  $b = \sqrt{100} = \measure{10}{cm}$.
  (Могло се и Питагорином теоремом: $b^2 = c^2 - a^2$)

  4. Израчунавање обима ($O$):
  $O = a + b + c$
  $O = 7{,}5 + 10 + 12{,}5 = \measure{30}{cm}$.
  Обим троугла је $\measure{30}{cm}$.
 \end{solution}
 \answerline

\end{questions}

% Уклоњено бодовање са краја документа, јер је пребачено у абстракт.

\end{document}
