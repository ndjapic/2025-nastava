\documentclass[11pt,a5paper,twoside,addpoints,noanswers]{exam} % задаци
%\documentclass[11pt,a5paper,twoside,addpoints,answers]{exam}   % решења

\usepackage[OT2]{fontenc}
\usepackage[utf8x]{inputenc}
\usepackage[serbian]{babel}
\usepackage{multicol}
\usepackage{amsmath,amssymb}
\usepackage{geometry}
\geometry{a5paper, margin=1.5cm}

\renewcommand{\solutiontitle}{\noindent\textbf{Решење:}\enspace}
\pointsinmargin
\pointname{}
\cfoot[]{Страна \thepage\ од \numpages}

\title{Иницијални тест за VIII разред}
\author{Градива VII разреда}
\date{септембар 2025.}

\pagestyle{headandfoot}
\runningheader{Иницијални тест -- VII разред}{}{варијанта 1}
\runningfooter{}{Страна \thepage\ од \numpages}{}

\begin{document}
\maketitle
\thispagestyle{headandfoot}

\ifprintanswers\else
\begin{flushleft}
\gradetable[v]\newpage
\end{flushleft}
\fi

\begin{questions}

% ---------- СТРАНА 1 ----------
\question[6]
Израчунај:
\begin{multicols}{3}
\begin{parts}
\part $15^2$
\part $\sqrt{144}$
\part $\sqrt{50}$
\end{parts}
\end{multicols}

\begin{solution}[\stretch 1]
\begin{parts}
\part $225$
\part $12$
\part $5\sqrt{2} \approx 7{,}07$
\end{parts}
\end{solution}

\question[6]
Упореди бројеве: $\sqrt{45}$ и $7$.

\begin{solution}[\stretch 1]
$\sqrt{45}\approx 6{,}7 < 7$.
\end{solution}

\ifprintanswers\else\newpage\fi

% ---------- СТРАНА 2 ----------
\question[6]
Израчунај и поједностави:
\[
(3\sqrt{5})^2 - 2\sqrt{25}.
\]

\begin{solution}[\stretch 1]
$(3\sqrt{5})^2=45$, \quad $2\sqrt{25}=10$,  
разлика $=35$.
\end{solution}

\question[8]
Реши једначину:
\[
x^2 = 49.
\]

\begin{solution}[\stretch 1]
$x=\pm 7$.
\end{solution}

\ifprintanswers\else\newpage\fi

% ---------- СТРАНА 3 ----------
\question[8]
Квадрат има страницу $a=12\,\text{cm}$. Израчунај дијагоналу и површину.

\begin{solution}[\stretch 2]
$d=a\sqrt{2}=12\sqrt{2}\approx 16{,}97\,\text{cm}$.  
$P=a^2=144\,\text{cm}^2$.
\end{solution}

\question[10]
Правоугли троугао има катете $9\,\text{cm}$ и $12\,\text{cm}$. Израчунај хипотенузу, обим и површину.

\begin{solution}[\stretch 2]
$c=\sqrt{9^2+12^2}=15\,\text{cm}$.  
$O=36\,\text{cm}$, \quad $P=54\,\text{cm}^2$.
\end{solution}

\ifprintanswers\else\newpage\fi

% ---------- СТРАНА 4 ----------
\question[10]
Круг има полупречник $r=7\,\text{cm}$. Израчунај обим и површину ($\pi\approx 3{,}14$).

\begin{solution}[\stretch 2]
$O=2\pi r \approx 43{,}96\,\text{cm}$.  
$P=\pi r^2 \approx 153{,}86\,\text{cm}^2$.
\end{solution}

\question[10]
(Текстуални задатак) Лења стаза кроз парк има облик правоуглог троугла. Катете су $60\,\text{m}$ и $80\,\text{m}$. Колико је краћа дијагонална стаза од збира катета?

\begin{solution}[\stretch 2]
$c=\sqrt{60^2+80^2}=100\,\text{m}$.  
Катете $=140\,\text{m}$.  
Разлика $=40\,\text{m}$.
\end{solution}

\end{questions}
\end{document}
