\documentclass[11pt,a5paper,twoside,addpoints,noanswers]{exam} % задаци
%\documentclass[11pt,a5paper,twoside,addpoints,answers]{exam}   % решења

\usepackage[OT2]{fontenc}
\usepackage[utf8x]{inputenc}
\usepackage[serbian]{babel}
\usepackage{multicol}
\usepackage{amsmath,amssymb}
\usepackage{geometry}
\geometry{a5paper, margin=1.5cm}

\renewcommand{\solutiontitle}{\noindent\textbf{Решење:}\enspace}
\pointsinmargin
\pointname{}
\cfoot[]{Страна \thepage\ од \numpages}

\title{Пробни иницијални тест за VIII разред (ИОП)}
\author{Олакшана варијанта}
\date{септембар 2025.}

\pagestyle{headandfoot}
\runningheader{Пробни иницијални тест (ИОП)}{}{једна варијанта}
\runningfooter{}{Страна \thepage\ од \numpages}{}

\begin{document}
\maketitle
\thispagestyle{headandfoot}

\ifprintanswers\else
\begin{flushleft}
\gradetable[v]\newpage
\end{flushleft}
\fi

\begin{questions}

% ---------- СТРАНА 1 ----------
\question[6]
Израчунај:
\begin{multicols}{3}
\begin{parts}
\part $25+37$
\part $120-85$
\part $7 \cdot 8$
\end{parts}
\end{multicols}

\begin{solution}[\stretch 1]
\begin{parts}
\part $62$
\part $35$
\part $56$
\end{parts}
\end{solution}

\question[6]
Израчунај:
\[
48 : 6 + 5 \cdot 4.
\]

\begin{solution}[\stretch 1]
$48:6=8$, $5\cdot4=20$, $8+20=28$.
\end{solution}

\ifprintanswers\else\newpage\fi

% ---------- СТРАНА 2 ----------
\question[6]
Реши једначину:
\[
x+7=15.
\]

\begin{solution}[\stretch 1]
$x=15-7=8$.
\end{solution}

\question[8]
Реши једначину:
\[
3x=21.
\]

\begin{solution}[\stretch 1]
$x=7$.
\end{solution}

\ifprintanswers\else\newpage\fi

% ---------- СТРАНА 3 ----------
\question[8]
Правоугаоник има странице $a=5\,\text{cm}$ и $b=9\,\text{cm}$.  
Израчунај обим и површину.

\begin{solution}[\stretch 2]
$O=2(a+b)=2\cdot 14=28\,\text{cm}$.  
$P=ab=45\,\text{cm}^2$.
\end{solution}

\question[10]
Квадрат има страницу $a=6\,\text{cm}$.  
Израчунај обим и површину.

\begin{solution}[\stretch 2]
$O=4a=24\,\text{cm}$.  
$P=a^2=36\,\text{cm}^2$.
\end{solution}

\ifprintanswers\else\newpage\fi

% ---------- СТРАНА 4 ----------
\question[10]
Круг има полупречник $r=7\,\text{cm}$.  
Израчунај приближно обим ($\pi\approx3{,}14$).

\begin{solution}[\stretch 2]
$O=2\pi r \approx 43{,}96\,\text{cm}$.
\end{solution}

\question[10]
(Текстуални задатак) Аутобус је прешао $120\,\text{km}$ за $2\,h$.  
Колика је просечна брзина?

\begin{solution}[\stretch 2]
$v=\tfrac{120}{2}=60\,\text{km/h}$.
\end{solution}

\end{questions}
\end{document}
