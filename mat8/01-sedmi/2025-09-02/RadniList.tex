\documentclass[11pt,a5paper,twoside,addpoints,noanswers]{exam} % задаци
%\documentclass[10pt,a5paper,twoside,addpoints,answers]{exam} % решења

\usepackage[OT2]{fontenc}
\usepackage[utf8x]{inputenc}
\usepackage[serbian]{babel}
\usepackage{multicol}
\usepackage{amssymb,amsmath}
\usepackage{graphicx}
\usepackage{geometry}
\geometry{a5paper, margin=1.5cm}

% Макро за јединице мере
\newcommand{\measure}[2]{#1\,\mathrm{#2}}

% Макро за варијанте
\newcommand{\variant}[4]{#1}

% exam подешавања
\renewcommand{\solutiontitle}{\noindent\textrm{Решење:}\enspace}
\pointsinmargin
\pointname{}
\hqword{Задатак:}
\hpgword{Страница:}
\hpword{Поени:}
\hsword{Остварено:}
\htword{Збир}
\vqword{Задатак:}
\vpgword{Страница:}
\vpword{Поени:}
\vsword{Остварено:}
\vtword{Збир}
\cellwidth{1em}
\cfoot[]{Страница \thepage\ од \numpages}
\addto{\captionsserbian}{\renewcommand{\abstractname}{Упутство}}

\title{Припремна настава}
\author{$\mathrm{VII}_\Box$ Примена Питагорине теореме. Степеновање
 \thanks{
  33 одлично,
  25 врло добро,
  17 добро,
   9 довољно.
 }
}
\date{Станишић, 13.\ август 2025.}

\pagestyle{headandfoot}
\runningheader{Примена Питагорине теореме. Степеновање}
	{}{варијанта \variant 1234}
\runningfooter{среда}
	{}{Страна \thepage\ од \numpages}

\begin{document}
\maketitle
\thispagestyle{headandfoot}

\ifprintanswers\else
\begin{flushleft}
\gradetable[v]\newpage
\end{flushleft}
\fi

\begin{questions}

\question %01.
Израчунај:
\begin{multicols}{3}
\begin{parts}
\part[1] $3^{\variant{4}{5}{6}{7}}$;
\part[1] $2^{\variant{10}{9}{8}{7}}$;
\part[1] $5^{\variant{3}{4}{5}{2}}$.
\end{parts}
\end{multicols}

\question %02.
Запиши у облику степена:
\begin{parts}
\part[1] $\variant
{7\cdot 7\cdot 7\cdot 7}
{0,\!4\cdot 0,\!4\cdot 0,\!4}
{(-3)\cdot (-3)\cdot (-3)\cdot (-3)\cdot (-3)}
{1,\!2\cdot 1,\!2\cdot 1,\!2\cdot 1,\!2}$;
\part[1] $\variant
{\sqrt{5}\cdot \sqrt{5}\cdot \sqrt{5}\cdot \sqrt{5}}
{\left(-\tfrac{2}{3}\right)\cdot\left(-\tfrac{2}{3}\right)\cdot\left(-\tfrac{2}{3}\right)}
{\dfrac{\sqrt{7}}{2}\cdot\dfrac{\sqrt{7}}{2}\cdot\dfrac{\sqrt{7}}{2}\cdot\dfrac{\sqrt{7}}{2}\cdot\dfrac{\sqrt{7}}{2}}
{(-4)\cdot (-4)\cdot (-4)}$.
\end{parts}

\question[3] %03.
Поређај бројеве од најмањег до највећег:
\[
\variant
 {3^4,\; 2^6,\; 7^2,\; 5^3}
 {2^8,\; 10^2,\; 3^5,\; 9^2}
 {11^2,\; 2^7,\; 4^4,\; 6^3}
 {2^9,\; 3^6,\; 5^4,\; 7^3}.
\]

\question %04.
Поједностави израз:
\begin{multicols}{2}
\begin{parts}
\part[1] $\dfrac{x^{12}}{x^7}$;
\part[2] $\dfrac{a^5\cdot a^8}{a^{11}}$.
\end{parts}
\end{multicols}

\question %05.
Израчунај вредност:
\begin{multicols}{2}
\begin{parts}
\part[1] $\dfrac{3^9}{3^7}$;
\part[2] $\dfrac{4^6}{2^8}$.
\end{parts}
\end{multicols}

\question[3] %06.
Одреди $x$:
\[
\variant{2^x=8^4}{(3^{2x})^2=3^{20}}{5^{x+1}=125}{7^{3x}=7^{12}}.
\]

\question[4] %07.
Напиши број \variant{1024}{3125}{81}{64}
као степен са основом \variant{2}{5}{3}{2},
а затим као степен са основом \variant{4}{25}{9}{8}.

% ---------------- Питагорина ----------------

\question[4] %08.
Једнакокраки троугао има основицу $\measure{\variant{12}{14}{10}{8}}{cm}$, а висину $\measure{\variant{9}{10}{8}{6}}{cm}$. Израчунај дужину крака, обим и површину троугла.

\question[4] %09.
Правоугаоник има странице $\measure{\variant{9}{12}{7}{15}}{cm}$ и $\measure{\variant{12}{5}{24}{8}}{cm}$. Израчунај дијагоналу, обим и површину.

\question %10.
\begin{parts}
\part[2] Квадрат има дијагоналу $\measure{\variant{8}{10}{12}{14}}{cm}$. Израчунај страницу и површину квадрата.
\part[3] Ромб има дијагонале $d_1=\measure{\variant{10}{15}{20}{12}}{cm}$ и $d_2=\measure{\variant{24}{7}{16}{18}}{cm}$. Израчунај страницу, површину и висину ромба.
\end{parts}

\question[5] %11.
Правоугли трапез има основице $\measure{\variant{14}{16}{12}{10}}{cm}$ и $\measure{\variant{8}{10}{6}{5}}{cm}$, а крак нормалан на основице $\measure{\variant{7}{6}{5}{8}}{cm}$. Израчунај други крак, обим и површину трапеза.

\end{questions}

\end{document}
