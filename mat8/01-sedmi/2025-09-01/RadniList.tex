\documentclass[11pt,a5paper,twoside,addpoints,noanswers]{exam} % задаци
%\documentclass[11pt,a5paper,twoside,addpoints,answers]{exam} % решења

\usepackage[OT2]{fontenc}
\usepackage[utf8x]{inputenc}
\usepackage[serbian]{babel}
\usepackage{multicol}
\usepackage{amssymb,amsmath}
\usepackage{graphicx}
\usepackage{geometry}
\geometry{a5paper, margin=1.5cm}

% Макро за јединице мере
\newcommand{\measure}[2]{\mathrm{#1\,#2}}

% Макро за варијанте (пример: \variant{a}{b}{c})
\newcommand{\variant}[3]{#1}

% Подешавања за exam
\renewcommand{\solutiontitle}{\noindent\textrm{Решење:}\enspace}
\pointsinmargin
\pointname{}
\hqword{Задатак:}
\hpgword{Страница:}
\hpword{Поени:}
\hsword{Остварено:}
\htword{Збир}
\cellwidth{1em}
\cfoot[]{Страница \thepage\ од \numpages}
\addto{\captionsserbian}{\renewcommand{\abstractname}{Упут{}ство}}

\title{$\mathrm{VIII}$ разред, варијанта \variant{1}{2}{3}}
\author{Обнављање градива — Реални бројеви и Питагорина теорема}
\date{Станишић, август 2025.}

\begin{document}
\maketitle
\thispagestyle{headandfoot}

\begin{center}
\gradetable[h]
\end{center}

\begin{questions}

\section{Лакши задаци}

\question[2]
Израчунај:
\begin{multicols}{3}
\begin{parts}
\part $\sqrt{\variant{36}{49}{64}}$
\ifprintanswers\else\answerline\fi
\part $\left|\variant{-12}{15}{-20}\right|$
\ifprintanswers\else\answerline\fi
\part $\dfrac{\variant{45}{56}{72}}{\variant{9}{7}{12}}$
\ifprintanswers\else\answerline\fi
\end{parts}
\end{multicols}
\begin{solution}[\stretch 1]
\begin{multicols}{3}
\part $\sqrt{\variant{36}{49}{64}}=\variant{6}{7}{8}$
\part $\left|\variant{-12}{15}{-20}\right|=\variant{12}{15}{20}$
\part $\dfrac{\variant{45}{56}{72}}{\variant{9}{7}{12}}=\variant{5}{8}{6}$
\end{multicols}
\end{solution}

\question[2]
Заокружи ирационалан број:  
$$
\variant{\sqrt{2},\;\;\tfrac{9}{3},\;\;7}{-5,\;\;\sqrt{18},\;\;2}{\tfrac{4}{5},\;\;\sqrt{5},\;\;1,\!2}
$$
\begin{solution}[\stretch 1]
Ирационалан број је:  
\variant{$\sqrt{2}$}{$\sqrt{18}$}{$\sqrt{5}$}
\end{solution}

\question[3]
Катете правоуглог троугла су дужине
$\measure{\variant{9}{12}{8}}{cm}$ и
$\measure{\variant{12}{5}{15}}{cm}$.  
Израчунај дужину хипотенузе.
\begin{solution}[\stretch 2]
\[
c=\sqrt{a^2+b^2}=\sqrt{\variant{9^2+12^2}{12^2+5^2}{8^2+15^2}}
=\sqrt{\variant{81+144}{144+25}{64+225}}
=\sqrt{\variant{225}{169}{289}}
=\measure{\variant{15}{13}{17}}{cm}
\]
\end{solution}

\section{Средњи задаци}

\question[4]
Допуни табелу:
$$
\begin{array}{|c|c|c|}\hline
x & x^2 & \sqrt{x^2}\\\hline
\variant{-6}{\tfrac{3}{2}}{-4} & & \\\hline
\variant{\tfrac{5}{3}}{-2}{1,\!7} & & \\\hline
\end{array}
$$
\begin{solution}[\stretch 2]
Први ред: $x^2=\variant{36}{\tfrac{9}{4}}{16}$, $\sqrt{x^2}=\variant{6}{\tfrac{3}{2}}{4}$  
Други ред: $x^2=\variant{\tfrac{25}{9}}{4}{2,\!89}$, $\sqrt{x^2}=\variant{\tfrac{5}{3}}{2}{1,\!7}$
\end{solution}

\question[4]
Хипотенуза правоуглог троугла је
$\measure{\variant{17}{20}{25}}{cm}$, а једна катета
$\measure{\variant{8}{12}{15}}{cm}$.  
Израчунај обим и површину троугла.
\begin{solution}[\stretch 4]
\begin{align*}
b &= \sqrt{c^2-a^2} = \sqrt{\variant{17^2-8^2}{20^2-12^2}{25^2-15^2}}
= \sqrt{\variant{289-64}{400-144}{625-225}}\\
&= \sqrt{\variant{225}{256}{400}}=\variant{15}{16}{20}\\
O &= a+b+c = \variant{8+15+17}{12+16+20}{15+20+25}
=\measure{\variant{40}{48}{60}}{cm}\\
P &= \tfrac{a\cdot b}{2}=\tfrac{\variant{8\cdot15}{12\cdot16}{15\cdot20}}{2}
=\measure{\variant{60}{96}{150}}{cm^2}
\end{align*}
\end{solution}

\question[3]
Израчунај:
\[
\dfrac{\sqrt{\variant{16}{49}{36}}-\sqrt{\variant{9}{25}{4}}}{\sqrt{\variant{25}{4}{49}}}
\]
\begin{solution}[\stretch 3]
Бројилац: $\variant{4-3}{7-5}{6-2}=\variant{1}{2}{4}$  
Именилац: $\variant{5}{2}{7}$  
Цео израз: $\dfrac{\variant{1}{2}{4}}{\variant{5}{2}{7}}=\variant{\tfrac{1}{5}}{1}{\tfrac{4}{7}}$
\end{solution}

\section{Тежи задаци}

\question[5]
Правоугли троугао има катете
$\measure{\variant{7}{15}{20}}{cm}$ и
$\measure{\variant{24}{8}{21}}{cm}$.  
Израчунај полупречник уписаног круга.
\begin{solution}[\stretch 3]
\begin{align*}
c &= \sqrt{a^2+b^2}=\sqrt{\variant{49+576}{225+64}{400+441}}\\
&=\sqrt{\variant{625}{289}{841}}=\variant{25}{17}{29}\\
r &= \tfrac{a+b-c}{2}=\tfrac{\variant{7+24-25}{15+8-17}{20+21-29}}{2}
=\tfrac{6}{2}=\measure{3}{cm}
\end{align*}
\end{solution}

\question[5]
(Домаћи) Израчунај површину троугла са страницама
$\measure{\variant{10}{7}{9}}{cm}$,
$\measure{\variant{13}{24}{12}}{cm}$,
$\measure{\variant{15}{25}{18}}{cm}$.
\begin{solution}[\stretch 4]
\begin{align*}
s &= \tfrac{a+b+c}{2}=\tfrac{\variant{10+13+15}{7+24+25}{9+12+18}}{2}
=\variant{19}{28}{19,5}\\
P &= \sqrt{s(s-a)(s-b)(s-c)}
=\sqrt{\variant{19\cdot9\cdot6\cdot4}{28\cdot21\cdot4\cdot3}{19,5\cdot10,5\cdot7,5\cdot1,5}}\\
P &= \measure{\variant{84}{84}{\approx56,1}}{cm^2}
\end{align*}
\end{solution}

\end{questions}

\end{document}
