\documentclass[11pt,a5paper,addpoints]{exam}

\usepackage[OT2]{fontenc}
\usepackage[utf8x]{inputenc}
\usepackage[serbian]{babel}
\usepackage{amsmath,amssymb}
\usepackage{graphicx}
\usepackage{geometry}
\geometry{a5paper, margin=1.5cm}

\renewcommand{\solutiontitle}{\noindent\textrm{Решење:}\enspace}
\pointsinmargin
\pointname{}
\hqword{Задатак:}
\hpgword{Страница:}
\hpword{Поени:}
\hsword{Остварено:}
\htword{Збир}
\cellwidth{1em}
\cfoot[]{Страница \thepage\ од \numpages}
\addto{\captionsserbian}{\renewcommand{\abstractname}{Упут{}ство}}

\pagestyle{headandfoot}
\firstpageheadrule
\firstpageheader{Математика 8}{}{Радни лист – обнављање}
\runningheader{Обнављање градива 7. разреда}{}{стр. \thepage\ од \numpages}
\runningheadrule
\runningfooter{}{}{}

\begin{document}

\begin{center}
\Large \textbf{Радни лист – обнављање градива (7. разред, I квартал)} \\
\smallskip
\small Време рада: 45 минута
\end{center}

\begin{questions}

% ======= Полиноми =======
\question[3]
Развиј и среди израз:
\[
(2x-3)(x+5).
\]
\begin{solution}[\stretch{2}]
$2x\cdot x+2x\cdot 5-3\cdot x-3\cdot 5=2x^2+10x-3x-15=2x^2+7x-15$.
\end{solution}

\question[3]
Примени формулу разлике квадрата:
\[
(5a-2)(5a+2).
\]
\begin{solution}[\stretch{1}]
$(5a)^2-2^2=25a^2-4$.
\end{solution}

% ======= Једначине и проценат =======
\question[4]
Реши једначину:
\[
3(x-4)=2x+1.
\]
\begin{solution}[\stretch{2}]
$3x-12=2x+1 \ \Rightarrow\ 3x-2x=1+12 \ \Rightarrow\ x=13$.
\end{solution}

\question[3]
Цена производа је снижена за $20\%$ и тада износи $1600$ динара. Колика је била почетна цена?
\begin{solution}[\stretch{2}]
После снижења остаје $80\%$ од цене: $0.8\cdot P=1600\Rightarrow P=\frac{1600}{0.8}=2000$ дин.
\end{solution}

% ======= Геометрија =======
\question[3]
У троуглу $\triangle ABC$ угао при врху $A$ износи $40^\circ$, угао при врху $B$ $85^\circ$. Израчунај угао при врху $C$.
\begin{solution}[\stretch{1}]
$\alpha+\beta+\gamma=180^\circ \ \Rightarrow \ \gamma=180^\circ-40^\circ-85^\circ=55^\circ$.
\end{solution}

\question[4]
Правоугаоник има странице $6$ cm и $8$ cm. Израчунај дужину дијагонале.
\begin{solution}[\stretch{2}]
По Питагори: $d=\sqrt{6^2+8^2}=\sqrt{36+64}=10$ cm.
\end{solution}

\question[4]
Квадрат има страницу $a=5$ cm. Израчунај површину уписане кружнице.
\begin{solution}[\stretch{2}]
Уписана кружница има пречник једнак страници квадрата, дакле $d=5$, $r=2.5$. Површина: $P=\pi r^2=\pi\cdot 2.5^2=6.25\pi\ \text{cm}^2$.
\end{solution}

\question[3]
Конструиши троугао $ABC$ ако је дато: $AB=5$ cm, $AC=6$ cm и $\angle A=60^\circ$.
\begin{solution}[\stretch{5}]
\begin{enumerate}
\item Нацртати дуж $AB=5$ cm.  
\item У тачки $A$ одмерити угао $60^\circ$ и на тој полуправој одмерити $AC=6$ cm.  
\item Спојити $B$ и $C$.  
\end{enumerate}
\end{solution}

\end{questions}

\end{document}
