\documentclass[10pt,a5paper,addpoints]{exam}

\usepackage{multicol}
\usepackage{amssymb,amsmath}
\usepackage{graphics}
\usepackage[OT2]{fontenc}
\usepackage[utf8x]{inputenc}
\usepackage[serbian]{babel}

\def\grupa#1#2#3#4{#4}
\title{$\mathrm{VI}$ разред, група \grupa 1234}
\author{Радни лист — обнављање градива \\ Саставио: ГПТ, 1. септембар 2025.}
\date{}
%\printanswers

\renewcommand{\solutiontitle}{\noindent\textrm{Решење:}\enspace}
\pointsinmargin
\pointname{}
\cellwidth{1em}
\cfoot[]{Страница \thepage\ од \numpages}

\def\abs|#1|{\left| #1 \right|}

\begin{document}

\maketitle
\thispagestyle{headandfoot}

\begin{questions}

% --- СТРАНА 1 ---
\question
Из скупа
\[
 M=\{\,4,\,-6,\,-18,\,3,\,0,\,-5,\,2\,\}
\]
издвој подскупове:
\begin{parts}
 \part[1] $A$ — скуп свих \emph{позитивних} целих бројева из $M$.
 \part[1] $B$ — скуп свих \emph{негативних} целих бројева из $M$.
\end{parts}

\begin{solution}[\stretch 2]
$A=\{4,3,2\}$, $B=\{-6,-18,-5\}$.
\end{solution}

\question
Израчунај:
\begin{multicols}{2}
\begin{parts}
 \part[1] $(-7)+(+5)$
 \part[1] $8+(-5)$
 \part[1] $(-6)+8$
 \part[1] $(-8)+(-2)$
\end{parts}
\end{multicols}

\begin{solution}[\stretch 2]
$(-7)+(+5)=-2,\quad 8+(-5)=3,\quad (-6)+8=2,\quad (-8)+(-2)=-10$.
\end{solution}

\ifprintanswers\else\newpage\fi

% --- СТРАНА 2 ---
\question
Израчунај:
\begin{parts}
 \part[1] $(-2)-(-7)$
 \part[1] $(+4)-(-6)$
 \part[1] $5-(-3)$
 \part[1] $(-2)-(+4)$
\end{parts}

\begin{solution}[\stretch 2]
$(-2)-(-7)=5,\quad (+4)-(-6)=10,\quad 5-(-3)=8,\quad (-2)-(+4)=-6$.
\end{solution}

\question
Множи и дели:
\begin{multicols}{2}
\begin{parts}
 \part[1] $(-3)\cdot(-5)$
 \part[1] $(-2)\cdot(+4)$
 \part[1] $(+2)\cdot(-3)$
 \part[1] $(-5)\cdot(-2)$
 \part[1] $(-16):(+4)$
 \part[1] $12:(-4)$
 \part[1] $(-15):(-3)$
 \part[1] $(-12):(-3)$
\end{parts}
\end{multicols}

\begin{solution}[\stretch 2]
$(-3)\cdot(-5)=15,\ (-2)\cdot4=-8,\ 2\cdot(-3)=-6,\ (-5)\cdot(-2)=10$\\
$(-16):4=-4,\ 12:(-4)=-3,\ (-15):(-3)=5,\ (-12):(-3)=4$.
\end{solution}

\question
Испитај да ли постоји троугао са датим дужинама:
\begin{parts}
 \part[2] $a=5{,}4$, $b=3{,}5$, $c=9{,}8$
 \part[2] Унутрашњи углови: $45^\circ, 65^\circ, 70^\circ$
\end{parts}

\begin{solution}[\stretch 2]
(а) $5{,}4+3{,}5=8{,}9<9{,}8$ — не постоји троугао.\\
(б) $45+65+70=180^\circ$ — могу бити у троуглу.
\end{solution}

\ifprintanswers\else\newpage\fi

% --- СТРАНА 3 ---
\question
Допуни реченице:
\begin{parts}
 \part[1] У једнакокраком троуглу једнаке странице називамо \underline{\hspace{3cm}}, а трећа страница је \underline{\hspace{3cm}}.
 \part[1] У правоуглом троуглу највећу страницу називамо \underline{\hspace{3cm}}, а две мање странице су \underline{\hspace{3cm}}.
 \part[1] Троугао чији је највећи угао туп називамо \underline{\hspace{3cm}}, а троугао који има три оштра угла је \underline{\hspace{3cm}}.
 \part[1] Троугао чије су све странице једнаке називамо \underline{\hspace{3cm}}, а троугао чије су две странице једнаке је \underline{\hspace{3cm}}.
\end{parts}

\begin{solution}[\stretch 2]
краци, основица; хипотенуза, катете; тупоугли троугао, оштроугли троугао; једнакостранични троугао, једнакокраки троугао
\end{solution}

\question
Дати су цели бројеви:
\[
11,12,-4,-21,-8,15,-10
\]
\begin{parts}
 \part[1] Нађи најмањи и највећи и израчунај збир и разлику.
 \part[1] Поређај од најмањег ка највећем.
\end{parts}

\begin{solution}[\stretch 2]
Најмањи $-21$, највећи $15$; збир $-6$, разлика $36$; редослед: $-21,-10,-8,-4,11,12,15$.
\end{solution}

\question[3]
Израчунај:
\[
-14 + 6\cdot(-3) - (-48):(-6)
\]

\begin{solution}[\stretch 2]
$6\cdot(-3)=-18$, $(-48):(-6)=8$; резултат $-14-18-8=-40$.
\end{solution}

\ifprintanswers\else\newpage\fi

% --- СТРАНА 4 ---
\question
У једнакокраком троуглу $\triangle ABC$ ($AC=BC$) угао при врху $C$ је $70^\circ$. Упореди дужину крака и основице.

\begin{solution}[\stretch 2]
Углови при основици $(180-70)/2=55^\circ$. Страна наспрам већег угла је већа → основица је дужа од крака.
\end{solution}

\question
Дате су две странице $a=11{,}5$ и $b=4{,}5$. Нађи границе за трећу страну и обим.

\begin{solution}[\stretch 2]
$|a-b|<c<a+b \implies 7{,}0<c<16{,}0$; обим $P=16{,}0+c\in(23{,}0,32{,}0)$.
\end{solution}

\end{questions}

\end{document}
