\documentclass[11pt,a5paper,addpoints]{exam}

\usepackage{multicol}
\usepackage{amssymb,amsmath}
\usepackage{graphicx}
\usepackage[OT2]{fontenc}
\usepackage[utf8x]{inputenc}
\usepackage[serbian]{babel}

\title{Други радни лист — обнављање градива $\mathrm{VI}$ разреда}
\author{Саставио: ГПТ, 2.\ септембар 2025.}
\date{}

\printanswers
\renewcommand{\solutiontitle}{\noindent\textrm{Решење:}\enspace}
\pointsinmargin
\pointname{}
\hqword{Задатак:}
\hpgword{Страница:}
\hpword{Поени:}
\hsword{Остварено:}
\htword{Збир}
\cellwidth{1em}
\cfoot[]{Страница \thepage\ од \numpages}

\begin{document}

\maketitle
\thispagestyle{headandfoot}

\begin{abstract}
Пре израде теста пажљиво прочитај текст задатка.
Обавезно наведи поступак израде.
Израда теста траје 40 минута.
\end{abstract}

\noindent \gradetable[h]

\begin{questions}

% --- СТРАНА 1 ---
\question[2]
Представи број $-\frac{11}{4}$ у облику мешовитог броја.

\begin{solution}[\stretch 2]
$-\frac{11}{4} = -2\frac{3}{4}$.  
Целобројни део је $-2$, остатак $3/4$.
\end{solution}

\question[2]
Који је већи број: $-\frac{7}{8}$ или $-\frac{5}{6}$? Објасни.

\begin{solution}[\stretch 2]
За негативне бројеве, онај са мањом апсолутном вредношћу је већи:  
$|-\frac{7}{8}| = 0,875$, $|-\frac{5}{6}| \approx 0,833$  
Закључак: већи је $-\frac{5}{6}$.
\end{solution}

\ifprintanswers\else\newpage\fi

% --- СТРАНА 2 ---
\question[3]
Поређај следеће бројеве од најмањег до највећег: $-\frac{3}{4},\ 1,25,\ -\frac{5}{2},\ 0,75$

\begin{solution}[\stretch 2]
Прво све у истом облику (десетичне):  
$-\frac{3}{4}=-0,75$, $1,25=1,25$, $-\frac{5}{2}=-2,5$, $0,75=0,75$  
Поређано: $-2,5,\ -0,75,\ 0,75,\ 1,25$
\end{solution}

\question[3]
Израчунај: $-\frac{2}{3} + \frac{5}{6} - \frac{1}{2}$

\begin{solution}[\stretch 2]
Заједнички именилац 6:  
$-\frac{2}{3}=-\frac{4}{6}$, $\frac{5}{6}=\frac{5}{6}$, $-\frac{1}{2}=-\frac{3}{6}$  
Саберемо: $(-4+5-3)/6 = -2/6=-\frac{1}{3}$
\end{solution}

\ifprintanswers\else\newpage\fi

% --- СТРАНА 3 ---
\question[3]
Израчунај: $2\frac{1}{3}-1\frac{5}{6}$

\begin{solution}[\stretch 2]
Претворимо у неправе разломке: $2\frac{1}{3}=7/3$, $1\frac{5}{6}=11/6$  
$7/3-11/6=14/6-11/6=3/6=1/2$
\end{solution}

\question[3]
Израчунај: $\frac{3}{4}\cdot(-2):\frac{1}{2}$

\begin{solution}[\stretch 2]
$\frac{3}{4}\cdot(-2)=-3/2$  
$-3/2 : 1/2 = -3/2 \cdot 2/1 = -3$
\end{solution}

\question[3]
Реши неједначину: $x+1,5<2,3$

\begin{solution}[\stretch 2]
Одузмемо $1,5$ са обе стране:  
$x<2,3-1,5\implies x<0,8$
\end{solution}

\ifprintanswers\else\newpage\fi

% --- СТРАНА 4 ---
\question[3]
Реши једначину: $x-1,8=-2,7$

\begin{solution}[\stretch 2]
Додајемо $1,8$ обе стране: $x=-2,7+1,8=-0,9$
\end{solution}

\question[3]
Реши једначину: $x+2,5=1,8$

\begin{solution}[\stretch 2]
Одузмемо $2,5$: $x=1,8-2,5=-0,7$
\end{solution}

\question[3]
Израчунај: $\frac{5}{6}-\frac{7}{12}+\frac{1}{4}$

\begin{solution}[\stretch 2]
Заједнички именилац 12: $\frac{5}{6}=10/12$, $-\frac{7}{12}=-7/12$, $\frac{1}{4}=3/12$  
$10-7+3=6/12=1/2$
\end{solution}

\end{questions}

\end{document}
