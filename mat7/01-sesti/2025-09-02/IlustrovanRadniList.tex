\documentclass[11pt,a5paper,addpoints]{exam}

\usepackage{multicol}
\usepackage{amssymb,amsmath}
\usepackage{graphicx}
\usepackage{tikz}
\usepackage[OT2]{fontenc}
\usepackage[utf8x]{inputenc}
\usepackage[serbian]{babel}

\title{Други радни лист — визуелно сабирање и одузимање разломака}
\author{Саставио: ГПТ, 2.\ септембар 2025.}
\date{}

\printanswers
\renewcommand{\solutiontitle}{\noindent\textrm{Решење:}\enspace}
\pointsinmargin
\pointname{}
\hqword{Задатак:}
\hpgword{Страница:}
\hpword{Поени:}
\hsword{Остварено:}
\htword{Збир}
\cellwidth{1em}
\cfoot[]{Страница \thepage\ од \numpages}

\begin{document}

\maketitle
\thispagestyle{headandfoot}

\begin{abstract}
Пре израде теста пажљиво прочитај текст задатка.
Обавезно наведи поступак израде.
Израда теста траје 40 минута.
\end{abstract}

\noindent \gradetable[h]

\begin{questions}

% --- СТРАНА 1 ---
\question[4]
Израчунај: $\frac{2}{3}+\frac{1}{6}-\frac{1}{2}$  
Прикажи сваки разломак визуелно са различитим бојама за делове.

\begin{center}
% \frac{2}{3} (зелена)
\begin{tikzpicture}
\fill[green!70] (0,0) rectangle (2,1); % два дела
\fill[gray!20] (2,0) rectangle (3,1); % остатак
\draw (0,0) rectangle (3,1);
\node at (1.5,1.2){$\frac{2}{3}$};
\end{tikzpicture}

\vspace{0.3cm}

% \frac{1}{6} (плава)
\begin{tikzpicture}
\fill[blue!70] (0,0) rectangle (1,1);
\fill[gray!20] (1,0) rectangle (6,1);
\draw (0,0) rectangle (6,1);
\node at (0.5,1.2){$\frac{1}{6}$};
\end{tikzpicture}

\vspace{0.3cm}

% \frac{1}{2} (црвена)
\begin{tikzpicture}
\fill[red!70] (0,0) rectangle (3,1); % 1/2 као 3/6
\fill[gray!20] (3,0) rectangle (6,1);
\draw (0,0) rectangle (6,1);
\node at (1.5,1.2){$\frac{1}{2}$};
\end{tikzpicture}
\end{center}

\begin{solution}[\stretch 2]
Претварамо све на именилац 6:  
$\frac{2}{3} = \frac{4}{6}$, $\frac{1}{6} = \frac{1}{6}$, $\frac{1}{2} = \frac{3}{6}$  

Саберемо: $4/6 + 1/6 - 3/6 = 2/6 = 1/3$  

Визуелно: зелено 4/6 + плаво 1/6 = 5/6, црвено 3/6 одузимамо = остатак 2/6 = 1/3.
\end{solution}

\ifprintanswers\else\newpage\fi

% --- СТРАНА 2 ---
\question[4]
Израчунај: $1\frac{1}{4} - \frac{3}{8}$  
Визуелно прикажи: зелено за целе и делове, плаво за одузимање.

\begin{center}
% 1 1/4
\begin{tikzpicture}
\fill[green!70] (0,0) rectangle (1,1); % цео 1
\fill[green!40] (1,0) rectangle (1.25,1); % 1/4
\draw (0,0) rectangle (1.25,1);
\node at (0.625,1.2){$1\frac{1}{4}$};
\end{tikzpicture}

\vspace{0.3cm}

% 3/8
\begin{tikzpicture}
\fill[blue!70] (0,0) rectangle (0.375,1); % 3/8 као 0.375
\fill[gray!20] (0.375,0) rectangle (1.25,1);
\draw (0,0) rectangle (1.25,1);
\node at (0.1875,1.2){$\frac{3}{8}$};
\end{tikzpicture}
\end{center}

\begin{solution}[\stretch 2]
Претварамо на именилац 8:  
$1\frac{1}{4}=1\frac{2}{8}$, одузимамо $3/8$  
$1 2/8 - 3/8 = 1 - 1/8 = 7/8$  
Визуелно: зелено 2/8 - плаво 3/8 = остатак 7/8 са целим 1.
\end{solution}

\ifprintanswers\else\newpage\fi

% --- СТРАНА 3 ---
\question[4]
Израчунај: $\frac{5}{6} - \frac{1}{3} + \frac{1}{2}$  
Прикажи различите боје за сваки разломак.

\begin{center}
\begin{tikzpicture}
\fill[green!70] (0,0) rectangle (5,1); % 5/6
\fill[blue!70] (0,0) rectangle (2,1); % 1/3 = 2/6
\draw (0,0) rectangle (6,1);
\node at (2.5,1.2){$5/6 - 1/3$};
\end{tikzpicture}

\vspace{0.3cm}

\begin{tikzpicture}
\fill[red!70] (0,0) rectangle (3,1); % 1/2 = 3/6
\draw (0,0) rectangle (6,1);
\node at (1.5,1.2){$+1/2$};
\end{tikzpicture}
\end{center}

\begin{solution}[\stretch 2]
Претварамо све на именилац 6:  
$5/6 - 2/6 + 3/6 = 6/6 = 1$
\end{solution}

\ifprintanswers\else\newpage\fi

% --- СТРАНА 4 ---
\question[3]
Реши једначину: $x + 2,5 = 3,8$

\begin{solution}[\stretch 2]
Одузмемо 2,5: $x = 3,8 - 2,5 = 1,3$
\end{solution}

\question[3]
Реши неједначину: $x - 1,2 > 0,5$

\begin{solution}[\stretch 2]
Додајемо 1,2 обе стране: $x > 1,7$
\end{solution}

\question[3]
Израчунај: $1\frac{3}{4} - \frac{5}{8}$

\begin{solution}[\stretch 2]
Претварамо на именилац 8: $1\frac{3}{4}=1\frac{6}{8}$  
$1\frac{6}{8} - 5/8 = 1\frac{1}{8}$
\end{solution}

\end{questions}

\end{document}
